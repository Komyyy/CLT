\chapter{The central limit theorem}

\section{The central limit theorem for real random variables}

\begin{lemma}\label{lem:deriv_charFun}
\leanok
\lean{iteratedDeriv_charFun}
\uses{def:charFun}
Let $X$ be a real random variable with characteristic function $\phi$ with $\mathbb{E}[\vert X \vert^n] < \infty$ and let $k \le n$.
Then the $k$\textsuperscript{th} derivative of $\phi$ is
$\phi^{(k)}(t) = \mathbb{E}[i X]^k e^{i t X}$ \: .
\end{lemma}

\begin{proof}\leanok
\uses{lem:charFun_contDiff}
\end{proof}

\begin{lemma}[Peano form of Taylor's theorem]\label{lem:taylor_peano}
\leanok
\lean{taylor_mean_remainder_peano}
For a $n$-times continuously differentiable function $f : \mathbb{R} \to E$, for any $x_0\in\mathbb{R}$,
\begin{align*}
f(x) - \sum_{i=0}^n\frac{f^{(i)}(x_0)}{i!}(x-x_0)^i = o((x - x_0)^k)
\: .
\end{align*}
\end{lemma}

\begin{proof}
This is proven in mathlib PR: \href{https://github.com/leanprover-community/mathlib4/pull/19796}{\#19796}.
\end{proof}

\begin{lemma}\label{lem:iIndepFun_iff_pi_map_eq_map}
\mathlibok
\lean{ProbabilityTheory.iIndepFun_iff_map_fun_eq_pi_map}
A finite collection of random variables are independent iff their joint law is the product of their respective laws.
\end{lemma}

\begin{proof}\leanok
\end{proof}

\begin{lemma}\label{lem:charFun_taylor}
\leanok
\lean{taylor_charFun}
Let $X$ be a real random variable with characteristic function $\phi$, with $\mathbb{E}[\vert X \vert^n] < \infty$. Then as $t \to 0$,
\begin{align*}
\phi(t) = \sum_{k=0}^n \frac{(it)^k \mathbb{E}[X^k]}{k!} + o(t^n)
\: .
\end{align*}
\end{lemma}

\begin{proof}\leanok
\uses{lem:deriv_charFun, lem:taylor_peano}
\end{proof}

\begin{lemma}\label{lem:tendsto_one_plus_div_pow_exp}
\leanok
\lean{tendsto_one_plus_div_cpow_cexp}
For $t\in\mathbb{C}$, $\lim_{n\to\infty}(1+t/n+o(1/n))^n=\exp(t)$ (where the little-$o$ term may be complex).
\end{lemma}

\begin{proof}\leanok
\end{proof}

\begin{theorem}[Central limit theorem]\label{thm:clt}
    \leanok
    \lean{ProbabilityTheory.central_limit}
    \uses{def:cvg_distribution}
Let $X_1, X_2, \ldots$ be i.i.d. real random variables with mean 0 and variance 1, and let $Z$ be a random variable with law $\mathcal N(0,1)$. Then
\begin{align*}
\frac{1}{\sqrt{n}}\sum_{k=1}^n X_k \xrightarrow{d} Z \: .
\end{align*}
\end{theorem}

\begin{proof}\uses{lem:charFun_smul, lem:charFun_add_of_indep, thm:charFun_tendsto_iff_measure_tendsto, lem:charFun_taylor, lem:gaussian_charFun, lem:tendsto_one_plus_div_pow_exp, lem:iIndepFun_iff_pi_map_eq_map}
\leanok
Let $S_n = \frac{1}{\sqrt{n}}\sum_{k=1}^n X_k$. Let $\phi$ be the characteristic function of $X_k$. By Lemma~\ref{lem:charFun_smul} and~\ref{lem:charFun_add_of_indep}, the characteristic function of $S_n$ is $\phi_n(t) = (\phi(n^{-1/2}t))^n$.

By Lemma \ref{lem:charFun_taylor},
\begin{align*}
\phi_n(t) = (\phi(n^{-1/2}t))^n = \left(1 - \frac{1}{2n}t^2 + o(\frac{1}{n})\right)^n \to_{n \to +\infty} e^{-t^2/2} \: .
\end{align*}

Since the r.h.s. is the characteristic function of $Z$, we conclude that $\frac{1}{\sqrt{n}}\sum_{k=1}^n X_k \xrightarrow{d} Z$ by Lemma~\ref{thm:charFun_tendsto_iff_measure_tendsto}.
\end{proof}


\section{The central limit theorem for random variables in $\mathbb{R}^d$}

\begin{theorem}[Cramér-Wold]\label{thm:cramer_wold}
    \uses{def:cvg_distribution}
Let $X, X_1, X_2, \ldots$ be random variables in $\mathbb{R}^d$.
Then $X_n \xrightarrow{d} X$ iff for every $a \in \mathbb{R}^d$, $\langle a, X_n \rangle \xrightarrow{d} \langle a, X \rangle$.
\end{theorem}

\begin{proof}
    \uses{thm:charFun_tendsto_iff_measure_tendsto}
By Theorem~\ref{thm:charFun_tendsto_iff_measure_tendsto}, the convergence in distribution of the random variables is equivalent to the convergence of their characteristic functions for each $a \in \mathbb{R}^d$, which for each such $a$ is equivalent to the convergence in distribution of the random variables $\langle a, X_n \rangle$ by the same theorem.
\end{proof}

\begin{theorem}[Central limit theorem in $\mathbb{R}^d$]\label{thm:multivariate_clt}
    %\leanok
    %\lean{ProbabilityTheory.multivariate_central_limit}
Let $X_1, X_2, \ldots$ be i.i.d. random variables in $\mathbb{R}^d$ with mean 0 and identity covariance matrix, and let $Z$ be a random variable with law $\mathcal N(0,I_d)$. Then
\begin{align*}
\frac{1}{\sqrt{n}}\sum_{k=1}^n X_k \xrightarrow{d} Z \: .
\end{align*}
\end{theorem}

\begin{proof}
\uses{thm:cramer_wold, thm:clt, lem:stdGaussian_finiteDimensional}

\end{proof}