\chapter{The characteristic function}

\section{Definition and first properties}

\begin{definition}[Characteristic function]\label{def:charFun}
Let $\mu$ be a measure on a real inner product space $E$. The characteristic function of $\mu$, denoted by $\hat{\mu}$, is the function $E \to \mathbb{C}$ defined by
\begin{align*}
\hat{\mu}(t) = \int_x e^{i \langle t, x \rangle} d\mu(x) \: .
\end{align*}
The characteristic function of a random variable $X$ is defined as the characteristic function of $\mathcal L(X)$.
\end{definition}

\begin{lemma}\label{lem:charFun_bounded}
\uses{def:charFun}
For all $t$, $\Vert\hat{\mu}(t)\Vert \le 1$.
\end{lemma}

\begin{lemma}\label{lem:charFun_continuous}
\uses{def:charFun}
The characteristic function is a continuous function.
\end{lemma}

\begin{lemma}\label{lem:charFun_neg}
\uses{def:charFun}
$\hat{\mu}(-t) = \overline{\hat{\mu}(t)}$.
\end{lemma}

\begin{lemma}\label{lem:charFun_smul}
\uses{def:charFun}
For $a \in \mathbb{R}$, the characteristic function of $a X$ is $t \mapsto \phi_X(at)$.
\end{lemma}

\begin{lemma}\label{lem:charFun_add_of_indep}
\uses{def:charFun}
If two random variables $X, Y : \Omega \to S$ are independent, then $X+Y$ has characteristic function $\phi_{X+Y} = \phi_X \phi_Y$. 
\end{lemma}

\section{Convergence of characteristic functions and weak convergence of measures}

\begin{lemma}\label{lem:starSubalgebra_expPoly}
The functions of the form $x \mapsto \sum_{k=1}^n a_k e^{i\langle t_k, x\rangle}$ for $n \in \mathbb{N}$, $a_1, \ldots, a_n \in \mathbb{R}$ and $t_1, \ldots, t_n \in E$ are a star-subalgebra of $\mathcal C_b(E, \mathbb{C})$ (the bounded continuous functions from $E$ to $\mathbb{C}$). 
\end{lemma}

Let $\mathcal M$ be the set of these functions, which we call exponential polynomials.

\begin{lemma}\label{lem:separating_expPoly}
\uses{lem:starSubalgebra_expPoly}
The star-subalgebra $\mathcal M$ separates points. (TODO: define this)
\end{lemma}

\begin{definition}\label{def:separating}
A set $\mathcal F$ of functions $E \to F$ separates points in $E$ if for all $x, y \in E$ with $x \ne y$, there exists $f \in \mathcal F$ with $f(x) \ne f(y)$.

A set $\mathcal F$ of functions $E \to F$ is separating in $\mathcal P(E)$ if for all probability measures $\mu, \nu$ on $E$ with $\mu \ne \nu$, there exists $f \in \mathcal F$ with $\mu[f] \ne \nu[f]$.
\end{definition}

\begin{theorem}\label{thm:separating_starSubalgebra}
\uses{def:separating}
Let $E$ be a complete separable pseudo-metric space. Let $\mathcal M \subseteq \mathcal C_b(E, \mathbb{C})$ be a star-subalgebra that separates points. Then $\mathcal M$ is separating in $\mathcal P(E)$.
\end{theorem}

\begin{lemma}\label{lem:ext_charFun}
Two probability measure are equal iff they have the same characteristic function.
\end{lemma}

\begin{proof}\uses{thm:separating_starSubalgebra}
\end{proof}

\begin{theorem}[Convergence of characteristic functions and weak convergence of measures]\label{thm:charFun_tendsto_iff_measure_tendsto}
Let $\mu, \mu_1, \mu_2, \ldots$ be probability measures with characteristic functions $\hat{\mu}, \hat{\mu}_1, \hat{\mu}_2, \ldots$. Then $\mu_n \xrightarrow{w} \mu$ iff for all $t$, $\hat{\mu}_n(t) \to \hat{\mu}(t)$.
\end{theorem}

\begin{proof}\uses{def:cvg_distribution}
For all $t$, $x \mapsto e^{i \langle t, x \rangle}$ is a bounded continuous function. Hence by the definition of convergence in distribution, $\mu_n \xrightarrow{w} \mu \implies \hat{\mu}_n(t) \to \hat{\mu}(t)$ for all $t$.

TODO reverse direction.
\end{proof}