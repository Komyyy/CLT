\chapter{The characteristic function}

\section{Definition and first properties}

\begin{definition}[Characteristic function]\label{def:charFun}
\lean{ProbabilityTheory.charFun} \leanok
Let $\mu$ be a measure on a real inner product space $E$. The characteristic function of $\mu$, denoted by $\hat{\mu}$, is the function $E \to \mathbb{C}$ defined by
\begin{align*}
\hat{\mu}(t) = \int_x e^{i \langle t, x \rangle} d\mu(x) \: .
\end{align*}
The characteristic function of a random variable $X$ is defined as the characteristic function of $\mathcal L(X)$.
\end{definition}


\begin{lemma}\label{lem:charFun_bounded}
\lean{ProbabilityTheory.norm_charFun_le_one} \leanok
\uses{def:charFun}
For all $t$, $\Vert\hat{\mu}(t)\Vert \le 1$.
\end{lemma}

\begin{proof}\leanok
\end{proof}


\begin{lemma}\label{lem:charFun_contDiff}
\leanok
\lean{contDiff_charFun}
\uses{def:charFun}
If $\mu[|X|^n] < \infty$ then the characteristic function of $\mu$ is continuously differentiable of order $n$.
\end{lemma}

\begin{proof}\leanok
\end{proof}


\begin{lemma}\label{lem:charFun_continuous}
\leanok
\lean{continuous_charFun}
\uses{def:charFun}
The characteristic function is a continuous function.
\end{lemma}

\begin{proof}\leanok
\uses{lem:charFun_contDiff}
\end{proof}


\begin{lemma}\label{lem:charFun_neg}
\lean{ProbabilityTheory.charFun_neg} \leanok
\uses{def:charFun}
$\hat{\mu}(-t) = \overline{\hat{\mu}(t)}$.
\end{lemma}

\begin{proof}\leanok
\end{proof}


\begin{lemma}\label{lem:charFun_smul}
\leanok
\lean{ProbabilityTheory.charFun_map_smul}
\uses{def:charFun}
For $a \in \mathbb{R}$ and $X$ a random variable with law $\mu$, the characteristic function of $a X$ is $t \mapsto \hat{\mu}(at)$.
\end{lemma}

\begin{proof}\leanok
\end{proof}


\begin{lemma}\label{lem:charFun_add_of_indep}
\leanok
\lean{ProbabilityTheory.charFun_conv}
\uses{def:charFun}
If two random variables $X, Y : \Omega \to S$ with laws $\mu$ and $\nu$ are independent, then $X+Y$ has characteristic function $\hat{\mu} \times \hat{\nu}$.
\end{lemma}

\begin{proof}\leanok
\end{proof}


\begin{lemma}\label{lem:integral_exp_I}
\leanok
\lean{ProbabilityTheory.integral_exp_Icc}
\begin{align*}
    \int_{-r}^r e^{i t} \, d t = 2 \sin r \: .
\end{align*}
\end{lemma}

\begin{proof}\leanok
\end{proof}


\begin{lemma}\label{lem:integral_charFun}
\uses{def:charFun}
\leanok
\lean{ProbabilityTheory.integral_charFun_Icc}
\begin{align*}
    \int_{-r}^r \hat{\mu}(t) dt = 2 r \int \frac{\sin(r x)}{r x} \, d\mu(x)
\end{align*}
\end{lemma}

\begin{proof}
\uses{lem:integral_exp_I}
\leanok
We will use Fubini's theorem to exchange the order of integration.
\begin{align*}
    \int_{-r}^r \hat{\mu}(t) \, dt
    &= \int_{-r}^r \int e^{i t x} \, d\mu(x) \, dt
    \\
    &= \int \left( \int_{-r}^r e^{i t x} \, dt \right) \, d\mu(x)
    \\
    &= \int 2 \frac{\sin(r x)}{x} \, d\mu(x)
    \\
    &= 2 r \int \frac{\sin(r x)}{r x} \, d\mu(x)
\end{align*}
\end{proof}


% \begin{lemma}\label{lem:charFun_bound_small}
% \leanok
% \lean{ProbabilityTheory.measure_Icc_le_charFun}
% \uses{def:charFun}
% For $\mu$ a probability measure on $\mathbb{R}$ and $r > 0$,
% \begin{align*}
% \mu [-r, r]
% &\le 2 r \int_{-1/r}^{1/r} |\hat{\mu}(t)| dt
% \: .
% \end{align*}
% \end{lemma}

% \begin{proof}
% TODO
% \end{proof}


\begin{lemma}\label{lem:charFun_bound_large}
\leanok
\lean{ProbabilityTheory.measure_abs_ge_le_charFun}
\uses{def:charFun}
For $\mu$ a probability measure on $\mathbb{R}$ and $r > 0$,
\begin{align*}
\mu \left\{x \mid |x| > r\right\}
&\le \frac{r}{2} \int_{-2/r}^{2/r} (1 - \hat{\mu}(t))dt
\: .
\end{align*}
\end{lemma}

\begin{proof}
\uses{lem:integral_charFun, lem:charFun_bounded}
\leanok
We will use Lemma~\ref{lem:integral_charFun} with $r$ replaced by $2/r$.
\begin{align*}
    \frac{r}{2} \int_{-2/r}^{2/r} (1 - \hat{\mu}(t))dt
    &= 2 - \frac{r}{2} \int_{-2/r}^{2/r} \hat{\mu}(t)dt
    \\
    &= 2 - 2 \int \frac{\sin(2 x / r)}{2 x / r} \, d\mu(x)
    \\
    &= 2 \int \left(1 - \frac{\sin(2 x / r)}{2 x / r}\right) \, d\mu(x)
    \\
    &\ge 2 \int_{|x| \ge r} \left(1 - \frac{\sin(2 x / r)}{2 x / r}\right) \, d\mu(x)
    \\
    &\ge \mu\{x \mid |x| > r\}
    \: .
\end{align*}
In the end we used that $\sin x \le x /2$ for $x \ge 2$.
\end{proof}


\begin{lemma}\label{lem:charFun_bound_inner}
\leanok
\lean{ProbabilityTheory.measure_abs_inner_ge_le_charFun}
\uses{def:charFun}
For $\mu$ a probability measure on an inner product space $E$ and $r > 0$, $a \in E$,
\begin{align*}
\mu \left\{x \mid \vert\langle a, x\rangle\vert > r\right\}
&\le \frac{r}{2} \int_{-2/r}^{2/r} (1 - \hat{\mu}(t \cdot a))dt
\: .
\end{align*}
\end{lemma}

\begin{proof}\uses{lem:charFun_bound_large}
\leanok
TODO (see code)
\end{proof}

% \begin{lemma}\label{lem:abs_sub_charFun}
% \leanok
% \lean{ProbabilityTheory.abs_sub_charFun_le}
% \uses{def:charFun}
% Let $X$ be a random variable with law $\mu$. Then for any $s, t$,
% \begin{align*}
% \vert \hat{\mu}(s) - \hat{\mu}(t) \vert
% \le 2 \mu\left[ \left\vert (s - t) X\right\vert \wedge 1\right]
% \: .
% \end{align*}
% \end{lemma}

% \begin{proof}
% \begin{align*}
% \vert \hat{\mu}(s) - \hat{\mu}(t) \vert
% &= \vert \mu\left[e^{isX} - e^{itX}\right] \vert
% \\
% &\le \mu\left[\vert e^{isX} - e^{itX} \vert\right]
% \\
% &= \mu\left[\vert 1 - e^{i(t - s)X} \vert\right]
% \le 2 \mu\left[ \left\vert (s - t) X\right\vert \wedge 1\right]
% \: .
% \end{align*}
% \end{proof}


\section{Convergence of characteristic functions and weak convergence of measures}


\begin{lemma}\label{lem:tendsto_M_of_tendsto_charFun}
\leanok
\lean{MeasureTheory.ProbabilityMeasure.tendsto_charPoly_of_tendsto_charFun}
Let $\mu, \mu_1, \mu_2, \ldots$ be probability measures on $\mathbb{R}^d$ with characteristic functions $\hat{\mu}, \hat{\mu}_1, \hat{\mu}_2, \ldots$ such that $\hat{\mu}_n \to \hat{\mu}$.
Then for all exponential polynomial $f \in \mathcal M$, $\mu_n[f] \to \nu[f]$.
\end{lemma}

\begin{proof}\uses{lem:starSubalgebra_expPoly}\leanok
\end{proof}


\begin{lemma}\label{lem:ext_charFun}
\leanok
\lean{MeasureTheory.ProbabilityMeasure.ext_of_charFun_eq}
Two finite measures on a complete separable metric space are equal iff they have the same characteristic function.
\end{lemma}

\begin{proof}\uses{thm:separating_starSubalgebra, lem:separates_points_expPoly}\leanok
See Mathlib PR \#19783.

The sub-algebra $\mathcal M$ of exponential polynomials is separating by Lemma~\ref{lem:separates_points_expPoly} and~\ref{thm:separating_starSubalgebra}. Equality of characteristic functions implies equality on $\mathcal M$, which then implies equality of the measures since $\mathcal M$ is separating.
\end{proof}


% \begin{lemma}\label{lem:tight_iff_charFun_equicontinuous}
% \leanok
% \lean{isTightMeasureSet_iff_equicontinuousAt_charFun}
% \uses{def:tight}
% A family of measures $(\mu_a)$ on a finite dimensional inner product space is tight iff the family of functions $(\hat{\mu}_a)$ is equicontinuous at 0.
% \end{lemma}

% \begin{proof}\uses{lem:charFun_bound_small, lem:charFun_bound_inner, lem:abs_sub_charFun}
% TODO: equicontinuous implies tight.

% For the other direction, by Lemma~\ref{lem:abs_sub_charFun}, for all $a$,
% \begin{align*}
%     \vert \hat{\mu}_a(s) - \hat{\mu}_a(t) \vert
%     \le 2 \mu_a\left[ \left\vert (s - t) X\right\vert \wedge 1\right]
%     \: .
% \end{align*}
% Since $(\mu_a)$ is tight, the right-hand side converges to 0 as $s - t \to 0$ uniformly in $a$.
% We obtain equicontinuity at 0 of the family $(\hat{\mu}_a)$.
% \end{proof}


\begin{lemma}\label{lem:tight_of_tendsto_charFun}
\leanok
\lean{isTightMeasureSet_of_tendsto_charFun}
\uses{def:tight}
Let $(\mu_n)_{n \in \mathbb{N}}$ be measures on $\mathbb{R}^d$ with characteristic functions $(\hat{\mu}_n)$. If $\hat{\mu}_n$ converges pointwise to a function $f$ which is continuous at 0, then $(\mu_n)$ is tight.
\end{lemma}

\begin{proof}\uses{lem:charFun_bound_inner, lem:isTightMeasureSet_iff_basis, lem:charFun_bounded}
\leanok
By Lemma~\ref{lem:charFun_bound_inner} and dominated convergence, for all $a \in \mathbb{R}^d$ and $r > 0$,
\begin{align*}
    \lim_{n \to +\infty} \mu_n \left\{x \mid |\langle a, x\rangle| \ge r\right\}
    &\le \lim_{n \to +\infty} \frac{r}{2} \int_{-2/r}^{2/r} (1 - \hat{\mu}_n(\langle a, t\rangle))dt
    \\
    &= \frac{r}{2} \int_{-2/r}^{2/r} (1 - f(\langle a, t\rangle))dt
\end{align*}
The right-hand side converges to 0 as $r \to +\infty$ by continuity of $f$ at 0.
This shows that the family is tight.
\end{proof}

Remark: the finite dimension is necessary.
Counterexample: $\ell^2$ with $\mu_n$ the law of $X_n = \sum_{k=1}^n \zeta_k e_k$ where $e_k = (0, \ldots, 0, 1, 0 \ldots)$ and the $\zeta_k$ are i.i.d. $\mathcal N(0,1)$.
Then $\hat{\mu}_n(t) \to e^{- \Vert t \Vert^2 / 2}$ for all $t \in \ell^2$ but $(\mu_n)$ is not tight.


\begin{theorem}[Convergence of characteristic functions and weak convergence of measures]
\label{thm:charFun_tendsto_iff_measure_tendsto}
\leanok
\lean{MeasureTheory.ProbabilityMeasure.tendsto_iff_tendsto_charFun}
\uses{def:weak_cvg_measure}
Let $\mu, \mu_1, \mu_2, \ldots$ be probability measures on $\mathbb{R}^d$ with characteristic functions $\hat{\mu}, \hat{\mu}_1, \hat{\mu}_2, \ldots$. Then $\mu_n \xrightarrow{w} \mu$ iff for all $t$, $\hat{\mu}_n(t) \to \hat{\mu}(t)$.
\end{theorem}

\begin{proof}\uses{thm:separating_starSubalgebra, lem:separates_points_expPoly, lem:tight_of_tendsto_charFun, lem:cvg_of_separating, lem:tendsto_M_of_tendsto_charFun, lem:charFun_continuous}
\leanok
For all $t$, $x \mapsto e^{i \langle t, x \rangle}$ is a bounded continuous function. Hence by the definition of convergence in distribution, $\mu_n \xrightarrow{w} \mu \implies \hat{\mu}_n(t) \to \hat{\mu}(t)$ for all $t$.

Let's now prove the other direction.
If the characteristic functions converge pointwise, we get from Lemma~\ref{lem:tight_of_tendsto_charFun} that the family of measures is tight.
Then apply Lemma~\ref{lem:cvg_of_separating} to the subalgebra $\mathcal M$: it suffices to show that $\mu_n[f] \to \mu[f]$ for all $f \in \mathcal M$. This is obtained with Lemma~\ref{lem:tendsto_M_of_tendsto_charFun}.
\end{proof}