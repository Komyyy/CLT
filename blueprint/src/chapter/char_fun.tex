\chapter{The characteristic function}

\section{Definition and first properties}

\begin{definition}[Characteristic function]\label{def:charFun}
\lean{ProbabilityTheory.charFun} \leanok
Let $\mu$ be a measure on a real inner product space $E$. The characteristic function of $\mu$, denoted by $\hat{\mu}$, is the function $E \to \mathbb{C}$ defined by
\begin{align*}
\hat{\mu}(t) = \int_x e^{i \langle t, x \rangle} d\mu(x) \: .
\end{align*}
The characteristic function of a random variable $X$ is defined as the characteristic function of $\mathcal L(X)$.
\end{definition}


\begin{lemma}\label{lem:charFun_bounded}
\lean{ProbabilityTheory.norm_charFun_le_one} \leanok
\uses{def:charFun}
For all $t$, $\Vert\hat{\mu}(t)\Vert \le 1$.
\end{lemma}

\begin{proof}\leanok
\end{proof}


\begin{lemma}\label{lem:charFun_continuous}
\uses{def:charFun}
The characteristic function is a continuous function.
\end{lemma}

\begin{proof}
\end{proof}


\begin{lemma}\label{lem:charFun_neg}
\lean{ProbabilityTheory.charFun_neg} \leanok
\uses{def:charFun}
$\hat{\mu}(-t) = \overline{\hat{\mu}(t)}$.
\end{lemma}

\begin{proof}\leanok
\end{proof}


\begin{lemma}\label{lem:charFun_smul}
\uses{def:charFun}
For $a \in \mathbb{R}$, the characteristic function of $a X$ is $t \mapsto \phi_X(at)$.
\end{lemma}

\begin{proof}
\end{proof}


\begin{lemma}\label{lem:charFun_add_of_indep}
\uses{def:charFun}
If two random variables $X, Y : \Omega \to S$ are independent, then $X+Y$ has characteristic function $\phi_{X+Y} = \phi_X \phi_Y$. 
\end{lemma}

\begin{proof}
\end{proof}


\begin{lemma}\label{lem:charFun_bound}
\uses{def:charFun}
For $\mu$ a probability measure on $\mathbb{R}$ and $r > 0$,
\begin{align*}
\mu \left\{x \mid |x| \ge r\right\}
&\le \frac{r}{2} \int_{-2/r}^{2/r} (1 - \hat{\mu}(t))dt
\: , \\
\mu [-r, r]
&\le 2 r \int_{-1/r}^{1/r} |\hat{\mu}(t)| dt
\: .
\end{align*}
\end{lemma}

\begin{proof}
TODO
\end{proof}


\begin{lemma}\label{lem:abs_sub_charFun}
\uses{def:charFun}
Let $X$ be a random variable with law $\mu$. Then for any $s, t$,
\begin{align*}
\vert \hat{\mu}(s) - \hat{\mu}(t) \vert
\le 2 \mathbb{E}\left[ \left\vert (s - t) X\right\vert \wedge 1\right]
\: .
\end{align*}
\end{lemma}

\begin{proof}
\begin{align*}
\vert \hat{\mu}(s) - \hat{\mu}(t) \vert
&= \vert \mathbb{E}\left[e^{isX} - e^{itX}\right] \vert
\\
&\le \mathbb{E}\left[\vert e^{isX} - e^{itX} \vert\right]
\\
&= \mathbb{E}\left[\vert 1 - e^{i(t - s)X} \vert\right]
\le 2 \mathbb{E}\left[ \left\vert (s - t) X\right\vert \wedge 1\right]
\: .
\end{align*}
\end{proof}


\begin{lemma}\label{lem:gaussian_charFun}
\leanok
\lean{ProbabilityTheory.charFun_gaussianReal}
\uses{def:charFun}
The Gaussian distribution $\mathcal N(m, \sigma^2)$ has characteristic function $\phi(t) = e^{itm - \sigma^2 t^2 /2}$.
\end{lemma}

\begin{proof}\leanok
\end{proof}



\section{Convergence of characteristic functions and weak convergence of measures}

\begin{lemma}\label{lem:tight_iff_charFun_equiContinuous}
\uses{def:tight}
A family of measures $(\mu_a)$ on a finite dimensional inner product space is tight iff the family of functions $(\hat{\mu}_a)$ is equi-continuous at 0.
\end{lemma}

\begin{proof}\uses{lem:charFun_bound, lem:abs_sub_charFun}
TODO
\end{proof}


\begin{lemma}\label{lem:tight_of_tendsto_charFun}
\uses{def:weak_cvg_measure}
Let $(\mu_n)_{n \in \mathbb{N}}$ be measures on $\mathbb{R}^d$ with characteristic functions $(\hat{\mu}_n)$. If $\hat{\mu}_n$ converges pointwise to a function $f$ which is continuous at 0, then $(\mu_n)$ is tight.
\end{lemma}

\begin{proof}\uses{lem:tight_iff_charFun_equiContinuous}
TODO
\end{proof}

Remark: the finite dimension is necessary. Counterexample: $\ell^2$ with $\mu_n$ the law of $X_n = \sum_{k=1}^n \zeta_k e_k$ where $e_k = (0, \ldots, 0, 1, 0 \ldots)$ and the $\zeta_k$ are i.i.d. $\mathcal N(0,1)$. Then $\hat{\mu}_n(t) \to e^{- \Vert t \Vert^2 / 2}$ for all $t \in \ell^2$ but $(\mu_n)$ is not tight (todo: why?).

\begin{lemma}\label{lem:eq_M_of_eq_charFun}
If two probability measures $\mu, \nu$ have same characteristic function, then for all exponential polynomial $f \in \mathcal M$, $\mu[f] = \nu[f]$.
\end{lemma}

\begin{proof}\uses{lem:starSubalgebra_expPoly}
\end{proof}


\begin{lemma}\label{lem:ext_charFun}
Two probability measures on a TODO space are equal iff they have the same characteristic function.
\end{lemma}

\begin{proof}\uses{lem:separating_expPoly, lem:eq_M_of_eq_charFun}
The sub-algebra $\mathcal M$ of exponential polynomials is separating by Lemma~\ref{lem:separating_expPoly}. Equality of characteristic functions implies equality on $\mathcal M$ by Lemma~\ref{lem:eq_M_of_eq_charFun}, which then implies equality of the measures since $\mathcal M$ is separating.
\end{proof}


\begin{theorem}[Convergence of characteristic functions and weak convergence of measures]\label{thm:charFun_tendsto_iff_measure_tendsto}
Let $\mu, \mu_1, \mu_2, \ldots$ be probability measures on $\mathbb{R}^d$ with characteristic functions $\hat{\mu}, \hat{\mu}_1, \hat{\mu}_2, \ldots$. Then $\mu_n \xrightarrow{w} \mu$ iff for all $t$, $\hat{\mu}_n(t) \to \hat{\mu}(t)$.
\end{theorem}

\begin{proof}\uses{lem:separating_expPoly, lem:tight_of_tendsto_charFun, lem:cvg_of_separating, lem:eq_M_of_eq_charFun}
For all $t$, $x \mapsto e^{i \langle t, x \rangle}$ is a bounded continuous function. Hence by the definition of convergence in distribution, $\mu_n \xrightarrow{w} \mu \implies \hat{\mu}_n(t) \to \hat{\mu}(t)$ for all $t$.

Let's now prove the other direction.
If the characteristic functions converge pointwise, we get from Lemma~\ref{lem:tight_of_tendsto_charFun} that the family of measures is tight. 
Then apply Lemma~\ref{lem:cvg_of_separating} to the subalbegra $\mathcal M$: it suffices to show that $\mu_n[f] \to \mu[f]$ for all $f \in \mathcal M$. This is obtained with Lemma~\ref{lem:eq_M_of_eq_charFun}.
\end{proof}