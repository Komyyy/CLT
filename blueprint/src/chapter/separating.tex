\chapter{Separating subalgebras}

\begin{definition}\label{def:separates_points}
\lean{Set.SeparatesPoints} \leanok
A set $\mathcal F$ of functions $E \to F$ separates points in $E$ if for all $x, y \in E$ with $x \ne y$, there exists $f \in \mathcal F$ with $f(x) \ne f(y)$.
\end{definition}

\begin{definition}\label{def:separating}
A set $\mathcal F$ of functions $E \to F$ is separating in $\mathcal P(E)$ if for all probability measures $\mu, \nu$ on $E$ with $\mu \ne \nu$, there exists $f \in \mathcal F$ with $\mu[f] \ne \nu[f]$.
\end{definition}

TODO: this is a special case of a definition about duality pairings. Two types $E$, $F$ and a bilinear function $B : E \times F \to \mathbb{k}$ form a pairing (in the def above $\mathcal P(E)$, $C_b(E, \mathbb{C})$ and the integral). A set $\mathcal E$ of $E$ is separating if for all $f, f' \in F$ with $f \ne f'$, there exists $x \in \mathcal E$ with $B(x, f) \ne B(x, f')$. 

\begin{lemma}\label{lem:bounded_continuous_separating}
\uses{def:separating}
In a Borel space $E$, the set $C_b(E, \mathbb{C})$ of bounded continuous functions from $E$ to $\mathbb{C}$ is separating in $\mathcal P(E)$.
\end{lemma}

\begin{proof}
The Mathlib lemma \texttt{MeasureTheory.FiniteMeasure.ext\_of\_forall\_lintegral\_eq} shows that $C_b(E, \mathbb{R}_+)$ is separating. Since $C_b(E, \mathbb{R}_{+}) \subseteq C_b(E, \mathbb{C})$, the larger set is also separating.
\end{proof}

\begin{lemma}\label{lem:exp_character}
\lean{Clt.expInnerMulI} \leanok
$x \mapsto (t \mapsto \exp(i \langle t, x \rangle))$ is a monoid homomorphism from $(E,+)$ to $C_b(E, \mathbb{C})$.
\end{lemma}

\begin{proof}\leanok
\end{proof}

\begin{lemma}\label{lem:starSubalgebra_expPoly}
\lean{Clt.expPoly} \leanok
The functions of the form $x \mapsto \sum_{k=1}^n a_k e^{i\langle t_k, x\rangle}$ for $n \in \mathbb{N}$, $a_1, \ldots, a_n \in \mathbb{R}$ and $t_1, \ldots, t_n \in E$ are a star-subalgebra of $C_b(E, \mathbb{C})$. 
\end{lemma}

\begin{proof}\uses{lem:exp_character} \leanok
The monoid homomorphism can be lifted to a homomorphism of algebras from \texttt{AddMonoidAlgebra ℝ E} to $C_b(E, \mathbb{C})$ using \texttt{AddMonoidAlgebra.lift}. $\mathcal M$ is the range of this homomorphism. The ``star'' part can be checked easily.
\end{proof}

Let $\mathcal M$ be the set of these functions, which we call exponential polynomials.

\begin{lemma}\label{lem:separating_expPoly}
\lean{Clt.expPoly_separatesPoints} \leanok
\uses{lem:starSubalgebra_expPoly}
The star-subalgebra $\mathcal M$ separates points.
\end{lemma}

\begin{proof}
TODO
\end{proof}

\begin{lemma}\label{lem:integral_restrict_compact}
Let $\mu$ be a measure on $E$ and $K$ be a compact set such that $\mu(K^c) \le \varepsilon$. Let $f \in C_b(E, \mathbb{C})$. Then
\begin{align*}
\left\Vert \mu[fe^{-\varepsilon \Vert f \Vert^2}] - \mu[f e^{-\varepsilon \Vert f \Vert^2} \mathbb{I}_K] \right\Vert
\le C \sqrt{\varepsilon} \: ,
\end{align*}
where $C = \sup_{x \ge 0} x e^{-x^2}$.
\end{lemma}

\begin{proof}
\begin{align*}
\left\Vert \mu[fe^{-\varepsilon \Vert f \Vert^2}] - \mu[f e^{-\varepsilon \Vert f \Vert^2} \mathbb{I}_K] \right\Vert
&= \left\Vert \mu[f e^{-\varepsilon \Vert f \Vert^2} \mathbb{I}_{K^c}] \right\Vert
\\
&\le \frac{1}{\sqrt{\varepsilon}} \mu \left[ \sqrt{\varepsilon} \Vert f \Vert e^{-\varepsilon \Vert f \Vert^2} \mathbb{I}_{K^c} \right]
\\
&\le C \sqrt{\varepsilon} \: ,
\end{align*}
\end{proof}

\begin{lemma}\label{lem:introduce_exponential}
For any $f \in C_b(E, \mathbb{C})$ and probability measures $\mu, \nu$,
\begin{align*}
\left\vert \mu[f] - \nu[f] \right\vert
= \lim_{\varepsilon \to 0} \left\vert \mu\left[f e^{-\varepsilon \Vert f \Vert^2} \right] - \nu\left[f e^{-\varepsilon \Vert f \Vert^2} \right] \right\vert
\: .
\end{align*}
\end{lemma}

\begin{proof}
\end{proof}

\begin{lemma}\label{lem:exponiential_M_eq_limit_M}
For $f \in \mathcal A$ a subalgebra of $C_b(E, \mathbb{C})$, $\varepsilon \ge 0$ and $n \in \mathbb{N}$, $f (1 - \frac{\varepsilon}{n} f)^n \in \mathcal A$. Furthermore, for a measure $\mu$,
\begin{align*}
\mu\left[f e^{-\varepsilon \Vert f \Vert^2}\right] = \lim_{n \to + \infty} \mu\left[f (1 - \frac{\varepsilon}{n} f)^n\right] \: .
\end{align*}
\end{lemma}

\begin{proof}
Dominated convergence.
\end{proof}

\begin{lemma}\label{lem:Cb_eq_of_separating}
Let $\mathcal A$ be a subalgebra of $C_b(E, \mathbb{C})$ that separates points. Let $\mu, \nu$ be two probability measures. If for all $f \in \mathcal A$, $\mu[f] = \nu[f]$, then for all $g \in C_b(E, \mathbb{C})$, $\mu[g] = \nu[g]$.
\end{lemma}

\begin{proof}\uses{lem:introduce_exponential, lem:integral_restrict_compact, lem:exponiential_M_eq_limit_M}
Let $g \in C_b(E, \mathbb{C})$ and $\varepsilon > 0$.

By Lemma~\ref{lem:introduce_exponential}, it suffices to show that $\lim_{\varepsilon \to 0} \left\vert \mu\left[g e^{-\varepsilon \Vert g \Vert^2} \right] - \nu\left[g e^{-\varepsilon \Vert g \Vert^2} \right] \right\vert = 0$.

The two measures are inner regular, hence there exists $K$ compact with $\mu(K^c) \le \varepsilon$ and $\nu(K^c) \le \varepsilon$.

By Lemma~\ref{lem:integral_restrict_compact}, it suffices to show that $\lim_{\varepsilon \to 0} \left\vert \mu\left[g e^{-\varepsilon \Vert g \Vert^2} \mathbb{I}_K \right] - \nu\left[g e^{-\varepsilon \Vert g \Vert^2} \mathbb{I}_K \right] \right\vert = 0$.

On $K$, the Stone-Weierstrass theorem \texttt{ContinuousMap.starSubalgebra\_topologicalClosure\_eq\_top\_of\_separatesPoints} gives that a subalgebra that separates points is dense. There exists then $f \in \mathcal A$ such that $\sup_{x \in K} \left\vert f(x) - g(x) \right\vert \le \varepsilon$.

TODO: we can bound $\left\vert \mu\left[g e^{-\varepsilon \Vert g \Vert^2} \mathbb{I}_K \right] - \mu\left[f e^{-\varepsilon \Vert f \Vert^2} \mathbb{I}_K \right] \right\vert$ by a function of $\varepsilon$ tending to 0. Same for $\nu$.

It remains to show that $\lim_{\varepsilon \to 0} \left\vert \mu\left[f e^{-\varepsilon \Vert f \Vert^2} \mathbb{I}_K \right] - \nu\left[f e^{-\varepsilon \Vert f \Vert^2} \mathbb{I}_K \right] \right\vert = 0$.

TODO. Use~\ref{lem:exponiential_M_eq_limit_M} to use that $\mu[f'] = \nu[f']$ for all $f' \in \mathcal A$.
\end{proof}

\begin{theorem}[Subalgebras separating points]\label{thm:separating_starSubalgebra}
\uses{def:separates_points}
Let $E$ be a complete separable pseudo-metric space. Let $\mathcal A \subseteq C_b(E, \mathbb{C})$ be a star-subalgebra that separates points. Then $\mathcal A$ is separating in $\mathcal P(E)$.
\end{theorem}

\begin{proof}\uses{lem:bounded_continuous_separating, lem:Cb_eq_of_separating}
Let $\mu$ and $\nu$ be two probability measures such that $\mu[f] = \nu[f]$ for all $f \in \mathcal A$. We want to prove that $\mu = \nu$. Since $C_b(E, \mathbb{C})$ is separating, it suffices to prove that for $g \in C_b(E, \mathbb{C})$, $\mu[g] = \nu[g]$. This is proved in Lemma~\ref{lem:Cb_eq_of_separating}
\end{proof}

\begin{lemma}\label{lem:cvg_of_separating}
Let $\mathcal A$ be a separating subalgebra of $C_b(E, \mathbb{C})$. Let $\mu, \mu_1, \mu_2, \ldots$ be probability measures, with $(\mu_n)$ tight. If for all $f \in \mathcal A$, $\mu_n[f] \to \mu[f]$, then $\mu_n \xrightarrow{w} \mu$.
\end{lemma}

\begin{proof}\uses{lem:introduce_exponential, lem:integral_restrict_compact, lem:exponiential_M_eq_limit_M}
To show convergence to $\mu$, it suffices to show that every subsequence has a subsequence that converges to $\mu$ (\texttt{Filter.tendsto_of_subseq_tendsto}).
Since the family is tight, it is relatively compact by Prokhorov's theorem. We get that each subsequence has a converging subsequence.
Let then $\mu'$ be a weak limit of $\mu_n$. We have $\mu_n[f] \to \mu'[f]$ and $\mu_n[f] \to \mu[f]$ for all $f \in \mathcal A$, hence $\mu'[f] = \mu[f]$ for all $f \in \mathcal A$. Since $\mathcal A$ is separating, $\mathcal \mu' = \mu$.

Alternative proof without Prokhorov's theorem: we can proceed similarly to the proof of Lemma~\ref{lem:Cb_eq_of_separating}, starting with: let $K$ be a compact such that $\mu(K^c) \le \varepsilon$ and $\mu_n(K^c) \le \varepsilon$ for all $n$, which exists since the family is tight. Etc. (check that this works)
\end{proof}
