\chapter{Gaussian measures}

\begin{definition}\label{def:gaussianReal}
\mathlibok
\lean{ProbabilityTheory.gaussianReal}
A measure on $\mathbb{R}$ is Gaussian if it is equal to $\mathcal{N}(m, \sigma^2)$ for some $m \in \mathbb{R}$ and $\sigma^2 \ge 0$, in which $\mathcal{N}(m, \sigma^2)$ is the measure absolutely continuous with respect to the Lebesgue measure with density $x \mapsto \frac{1}{\sqrt{2 \pi \sigma^2}}e^{- \frac{1}{2\sigma^2}(x - m)^2}$ if $\sigma^2>0$ and the Dirac probability measure at $m$ if $\sigma^2 = 0$.
\end{definition}


\begin{lemma}\label{lem:isProbabilityMeasure_gaussianReal}
\mathlibok
\uses{def:gaussianReal}
A real Gaussian measure is a probability measure.
\end{lemma}

\begin{proof}\leanok
\end{proof}


\begin{lemma}\label{lem:gaussian_charFun}
\leanok
\lean{ProbabilityTheory.charFun_gaussianReal}
\uses{def:charFun, def:gaussianReal}
The Gaussian distribution $\mathcal N(m, \sigma^2)$ has characteristic function $\phi(t) = e^{itm - \sigma^2 t^2 /2}$.
\end{lemma}

\begin{proof}\leanok
\end{proof}


\begin{lemma}\label{lem:add_gaussianReal}
\uses{def:gaussianReal}
The sum of two independent real Gaussian random variables is Gaussian.
\end{lemma}

\begin{proof}
\uses{lem:gaussian_charFun, lem:charFun_add_of_indep}

\end{proof}


\begin{definition}\label{def:isGaussian}
\uses{def:gaussianReal}
\leanok
\lean{ProbabilityTheory.IsGaussian}
A Borel measure $\mu$ on a separable Banach space $E$ is said to be Gaussian if for all continuous linear maps $\ell : E \to \mathbb{R}$, the pushforward $\ell_*\mu$ is a real Gaussian measure.
\end{definition}

\begin{lemma}\label{lem:isProbabilityMeasure_isGaussian}
\leanok
\uses{def:isGaussian}
A Gaussian measure is a probability measure.
\end{lemma}

\begin{proof}\leanok
\uses{lem:isProbabilityMeasure_gaussianReal}
\end{proof}


\begin{lemma}\label{lem:isGaussian_gaussianReal}
\leanok
\lean{ProbabilityTheory.isGaussian_gaussianReal}
\uses{def:gaussianReal, def:isGaussian}
The real Gaussian measures are Gaussian measures in the sense of Definition~\ref{def:isGaussian}.
\end{lemma}

\begin{proof}\leanok

\end{proof}


\begin{lemma}\label{lem:isGaussian_add}
\leanok
\lean{ProbabilityTheory.isGaussian_map_prod_add}
\uses{def:isGaussian}
The sum of two independent Gaussian random variables is Gaussian.
\end{lemma}

\begin{proof}\leanok
\uses{lem:add_gaussianReal,lem:isProbabilityMeasure_isGaussian}

\end{proof}


\begin{lemma}\label{lem:stdGaussian_finiteDimensional}
\uses{def:isGaussian}
Let $E$ be a finite dimensional real inner product space and let $b_1, \ldots, b_d$ be an orthonormal basis of $E$.
Let $X_1, \ldots, X_d$ be independent standard Gaussian random variables on $\mathbb{R}$.
Then the law of $X_1 b_1 + \ldots + X_d b_d$ is a Gaussian measure on $E$.
\end{lemma}

\begin{proof}
\uses{lem:isGaussian_add}

\end{proof}